\documentclass[a5paper,16pt]{book} % В. А. Зорич Математический анализ Часть 1 Издание четвертое, исправленное 2002 год, страница 221
\usepackage[left=10mm, top=15mm, right=20mm, bottom=15mm, nohead, nofoot]{geometry}
\usepackage{pscyr}
\usepackage[russian]{babel}
\usepackage{wasysym}
\usepackage{amssymb}
\usepackage{setspace}
\usepackage{amsfonts}
\usepackage{tikz}

\setlength{\headheight}{0mm}
\setlength{\headsep}{0mm}

\author{Матвей Попов}
\date{22 февраля 2021}

\setcounter{page}{221}

\begin{document}
	\begin{center}
		\begin{spacing}{0.3}
			§1. ДИФФЕРЕНЦИРУЕМАЯ ФУНКЦИЯ
			\noindent\rule{\textwidth}{1pt}
		\end{spacing}
	\end{center}
	Найдем касательную к графику в точке (0,0). Поскольку
	\begin{center}
	$f'(0) = \lim\limits_{x\to 0}\frac{x^2\sin\frac{1}{x} - 0}{x - 0} = \lim\limits_{x\to 0}x\sin\frac{1}{x} = 0$,
	\end{center}  то касательная имеет уравнение $y - 0 = 0 \cdot (x - 0)$, или просто $y = 0$.
	
	Таким образом, в нашем примере касательная совпадает с осью $Ox$,с которой график имеет бесконечное количество точек пересечения в любой окрестности точки касания.
	\begin{spacing}{2}
		
	\end{spacing}
	\begin{tikzpicture}
		\draw (4.8,0) -- (8.8,0);
		\draw (6.8,0) -- (6.8,2);
		\draw (6.8,0) -- (8.3,1.5);
		\draw (6.8,0) -- (5.3,1.5);
		\draw (6.6,1.8) node {$y$};
		\draw (6.8,-0.2) node {0};
		\draw (6.8,-0.8) node {Рис. 18.};
		\draw (8.6,-0.2) node {$x$};
		\draw (8.2,0.6) node {$y = |x|$};
		\draw (0.8,2) node {В силу определения дифференцируе-};
		\draw (0.6,1.6) node {мости функции $f:E \to \mathbb{R}$ в точке $x_0 \in$};
		\draw (-1.7,1.2) node {$\in E$ имеем};
		\draw (0.6,0) node {$f(x) - f(x_0) = A(x_0)(x - x_0) + o(x - x_0)$};
		\draw (2.3,-0.6) node {при $x \to x_0$, $x \in E$.};
	\end{tikzpicture}
	\\Поскольку правая часть этого равенства стремится к нулю при $x\to x_0$, $x \in E$,\, то\, $\lim\limits_{E\ni x\to x_0}f(x)\; =\; f(x_0)$,\, так\, что\, дифференцируемая\, в\, точке\\ функция обязана быть непрерывной в этой точке.
	
	Покажем, что обратное, конечно, не всегда имеет место.
	
	\textbf{Пример 8.} Пусть $f(x) = |x|$ (рис. 18). Тогда в точке $x_0 = 0$
	\begin{center}
		$\lim\limits_{x\to x_0-0}\frac{f(x) - f(x_0)}{x - x_0} = \lim\limits_{x\to -0}\frac{|x| - 0}{x - 0} = \lim\limits_{x\to -0}\frac{-x}{x} = -1$,
		
		$\lim\limits_{x\to x_0+0}\frac{f(x) - f(x_0)}{x - x_0} = \lim\limits_{x\to +0}\frac{|x| - 0}{x - 0} = \lim\limits_{x\to +0}\frac{x}{x} = 1$.
	\end{center}
	Следовательно, в этой точке функция не имеет производной, а значит, и не дифференцируема в этой точке.
	
	\textbf{Пример 9.} Покажем, что $e ^{x+h} - e ^{x} = e ^{x}h + o(h)$ при $h \to 0$.
	
	Таким\, образом,\, функция\, $\exp (x)\; =\; e^x$ дифференцируема,\, причем \\$d\exp (x)h = \exp (x)h$, или $de^x = e^xdx$, и тем самым $\exp 'x = \exp x$, или $\frac{de^x}{dx} = e^x$.
	
	$\blacktriangleleft$
	$e^{x+h} - e^x = e^x(e^h - 1) = e^x(h + o(h)) = e^xh + o(h)$.
	
	Мы воспользовались полученной в примере 39, гл. III, §2, п.4 формулой $e^h - 1 = h + o(h)$ при $h \to 0$.
	$\blacktriangleright$
\end{document}
