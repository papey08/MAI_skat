\documentclass[12pt]{article}

\usepackage{fullpage}
\usepackage{multicol,multirow}
\usepackage{tabularx}
\usepackage{ulem}
\usepackage[utf8]{inputenc}
\usepackage[russian]{babel}

\usepackage{pgfplots}
% \pgfplotsset{compat=1.9}


% Оригиналный шаблон: http://k806.ru/dalabs/da-report-template-2012.tex

\begin{document}

\section*{Лабораторная работа №\,2 по курсу дискрeтного анализа: Сбаласированные деревья}

Выполнил студент группы М8О-208Б-20 МАИ \textit{Попов Матвей}.

\subsection*{Условие}

% Кратко описывается задача: 
\begin{enumerate}
\item Необходимо создать программную библиотеку, реализующую указанную структуру данных, 
на основе которой разработать программу-словарь. В словаре каждому ключу, представляющему 
из себя регистронезависимую последовательность букв английского алфавита длиной не более 
256 символов, поставлен в соответствие некоторый номер, от $0$ до $2^{64} - 1$. Разным словам 
может быть поставлен в соответствие один и тот же номер.
\item Вариант задания: B-дерево.
\item Операции: вставка, поиск, удаление элемента, сохранение словаря в файл, загрузка словаря 
из файла.
\end{enumerate}

\subsection*{Метод решения}

Было принято решение реализовать отдельный класс для дерева и отдельную структуру для узла 
дерева, чтобы класс дерева и все методы для работы с ним представляли из себя обёртку для 
идентичных функций над корневым узлом.

\subsection*{Описание программы}

Программа состоит из одного файла, в котором помимо функции main находятся два пространства 
имён: для узла дерева и самого дерева. В этих пространствах имён содержатся функции для 
вставки пары в дерево, поиска, удаления, сохранения дерева в файл и загрузки дерева из файла.

\subsection*{Дневник отладки}

\begin{enumerate}
\item Программа загружена на чекер без функций сохранения и загрузки, чтобы убедиться в 
работоспособности вставки, поиска и удаления. Выяснилось, что функция удаления написана неверно.
\item Переписана функция удаления, из-за многочисленных Segmentation Fault было принято решение 
отказаться от разбиения функции удаления на функции для обработки частных случаев.
\item Написаны функции сохранения и загрузки, однако программа не проходит тест №7.
\item Функции сохранения и загрузки переписаны дважды, программа проходит все тесты чекера.
\end{enumerate}


\subsection*{Тест производительности}

Во всех графиках по оси Y отложено время выполнения (в миллисекундах), по оси X — количество 
команд.

\subsubsection*{Вставка}

\begin{tikzpicture}
    \begin{axis} [
        legend pos = north west,
        ymin = 0
    ]
    \legend{
        IBTree::TBTree,
        std::map
    };
    \addplot coordinates {
        (1000,153) (10000,2156) (100000,22531)
    };
    \addplot coordinates {
        (1000,160) (10000,1390) (100000,15703)
    };
    \end{axis}
\end{tikzpicture}

\begin{tabular}{ | 1 | 1 | 1 | }
    \hline
        Кол-во строк & std::map & IBTree::TBTree \\ \hline
        1000 & 160 & 153 \\
        10000 & 1390 & 2156 \\
        100000 & 15703 & 22531 \\
    \hline
\end{tabular}

\subsubsection*{Поиск}

\begin{tikzpicture}
    \begin{axis} [
        legend pos = north west,
        ymin = 0
    ]
    \legend{
        IBTree::TBTree,
        std::map
    };
    \addplot coordinates {
        (1000,8) (10000,75) (100000,793)
    };
    \addplot coordinates {
        (1000,158) (10000,1449) (100000,15218)
    };
    \end{axis}
\end{tikzpicture}

\begin{tabular}{ | 1 | 1 | 1 | }
    \hline
        Кол-во строк & std::map & IBTree::TBTree \\ \hline
        1000 & 158 & 8 \\
        10000 & 1449 & 75 \\
        100000 & 15218 & 793 \\
    \hline
\end{tabular}

\subsubsection*{Удаление}

\begin{tikzpicture}
    \begin{axis} [
        legend pos = north west,
        ymin = 0
    ]
    \legend{
        IBTree::TBTree,
        std::map
    };
    \addplot coordinates {
        (1000,108) (10000,1327) (100000,12553)
    };
    \addplot coordinates {
        (1000,146) (10000,1561) (100000,15832)
    };
    \end{axis}
\end{tikzpicture}

\begin{tabular}{ | 1 | 1 | 1 | }
    \hline
        Кол-во строк & std::map & IBTree::TBTree \\ \hline
        1000 & 146 & 108 \\
        10000 & 1561 & 1327 \\
        100000 & 15832 & 12553 \\
    \hline
\end{tabular}       



Таким образом, операции над B-деревом имеют ту же сложность, что и над std::map, вставка и удаление 
по времени работают примерно одинаково, но поиск в B-дереве происходит гораздо быстрее. Это может быть 
связано с тем, что тестирование проходило над B-деревом со степенью 500, узлы такого B-дерева занимают
больше памяти, однако чтобы найти какую-либо пару, в среднем понадобится меньше итераций.


\subsection*{Недочёты}

После многочисленных правок код программы получился практически нечитаемым, а размер файла 
составил 
670
строк.

\subsection*{Выводы}

B-дерево — очень полезная структура для хранения данных, так как операции вставки, поиска и 
удаления элементов в среднем работают быстрее, чем в несбалансированном бинарном дереве 
поиска. Однако недостатком этой структуры является то, что реализации этих функций очень сложны.

\end{document}
