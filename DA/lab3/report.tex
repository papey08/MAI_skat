\documentclass[12pt]{article}

\usepackage{fullpage}
\usepackage{multicol,multirow}
\usepackage{tabularx}
\usepackage{ulem}
\usepackage[utf8]{inputenc}
\usepackage[russian]{babel}
% \usepackage{array}

\usepackage{listings}
\usepackage{xcolor}

\usepackage{pgfplots}
% \pgfplotsset{compat=1.9}


% Оригиналный шаблон: http://k806.ru/dalabs/da-report-template-2012.tex

\begin{document}

\section*{Лабораторная работа №\,3 по курсу дискрeтного анализа: Сбаласированные деревья}

Выполнил студент группы М8О-208Б-20 МАИ \textit{Попов Матвей}.

\subsection*{Условие}

% Кратко описывается задача: 
Для реализации словаря из предыдущей лабораторной работы,
необходимо провести исследование скорости выполнения и потребления
оперативной памяти.

\subsection*{gprof}

\subsubsection*{Основная информация}

Утилита gprof позволяет измерить время работы всех функций, методов и операторов
программы, количество их вызовов и долю от общего времени работы программы в 
процентах.

\subsubsection*{Команды для работы с утилитой}

Сначала скомпилируем исходную программу с ключом $-pg$:

\begin{lstlisting}
    g++ lab.cpp -pg -o lab
\end{lstlisting}

Затем запустим программу, передав ей на ввод файл $test.txt$, в котором 
содержится по 5000 команд на вставку, поиск и удаление:

\begin{lstlisting}
    ./lab <test.txt >out.txt
\end{lstlisting}

Выполнив эту команду, заметим, что кроме файла $out.txt$, в котором содержатся 
результаты выполнения команд из $test.txt$, появился файл $gmon.out$, в котором
содержится вся информация, предоставляемая утилитой gprof. Чтобы получить 
текстовый файл, выполним следующую команду:

\begin{lstlisting}
    gprof lab gmon.out > profile-data.txt
\end{lstlisting}

Таким образом, выполнив 3 простые команды, получили текстовый файл с подробной 
информацией о времени работы и вызовах всех функций и операторов, которые 
использовались в программе.

\subsubsection*{Результат работы утилиты}

Ниже приведена таблица, в которую перенесены данные из файла $profile-data.txt$,
полученного с помощью утилиты gprof.

\begin{tabular}{ | c | c | c | c | }
    \hline
        \% time & self seconds & calls & name \\ \hline
        93.33 & 2.09 & 3664224 & IPair::TPair::operator= \\
        1.34 & 0.03 & 33982 & Clear(char*) \\
        1.34 & 0.03 & 25233 & IPair::TPair::TPair(IPair::TPair const\&) \\
        0.89 & 0.02 & 330720 & IPair::operator< \\
        0.89 & 0.02 & 20000 & ToLower(char*, IPair::TPair\&) \\
        0.45 & 0.01 & 54943 & IBTree::BinarySearch \\
        0.45 & 0.01 & 42109 & IPair::operator== \\
    \hline
\end{tabular}

Все остальные функции, по данным результатам измерений утилиты gprof, 
работали примерно 0 секунд, поэтому в таблицу внесены не были. Из полученных 
данных следует, что большая часть времени работы программы тратится на операцию 
копирования пары «ключ-значение», это может быть связано с использованием 
объектов класса $vector$ в узлах дерева.

\subsection*{valgrind}

Valgrind является самым распространённым инструментом для отслеживания утечек 
памяти и других ошибок, связанных с памятью. Чтобы проверить программу $lab$ 
на проблемы с памятью, выполним следующую команду:

\begin{lstlisting}
    valgrind ./lab <test.txt >out.txt
\end{lstlisting}

В результате выполнения этой команды получаем следующее сообщение:

\begin{lstlisting}
    ==8733== Process terminating with default action of signal 2 (SIGINT)
    ==8733==    at 0x4B66075: write (write.c:26)
    ==8733==    by 0x492777D: std::__basic_file<char>
    ::xsputn(char const*, long) (in /usr/lib
    /x86_64-linux-gnu/libstdc++.so.6.0.28)
    ==8733==    by 0x4966EA0: std::basic_filebuf<char, 
    std::char_traits<char> >::_M_convert_to_external
    (char*, long) (in /usr/lib/x86_64-linux-gnu/libstdc++.so.6.0.28)
    ==8733==    by 0x49672FB: std::basic_filebuf<char,
     std::char_traits<char> >::overflow(int) (in /usr
     /lib/x86_64-linux-gnu/libstdc++.so.6.0.28)
    ==8733==    by 0x496505C: std::basic_filebuf<char,
     std::char_traits<char> >::sync() (in /usr/lib
     /x86_64-linux-gnu/libstdc++.so.6.0.28)
    ==8733==    by 0x498D7A2: std::ostream::flush() 
    (in /usr/lib/x86_64-linux-gnu/libstdc++.so.6.0.28)
    ==8733==    by 0x10CF92: main (in /mnt/c/Home/Prog/DA/lab2/lab)
    ==8733== 
    ==8733== HEAP SUMMARY:
    ==8733==     in use at exit: 266,288 bytes in 9 blocks
    ==8733==   total heap usage: 27 allocs, 18 frees, 
    482,064 bytes allocated
    ==8733== 
    ==8733== LEAK SUMMARY:
    ==8733==    definitely lost: 0 bytes in 0 blocks
    ==8733==    indirectly lost: 0 bytes in 0 blocks
    ==8733==      possibly lost: 0 bytes in 0 blocks
    ==8733==    still reachable: 266,288 bytes in 9 blocks
    ==8733==         suppressed: 0 bytes in 0 blocks
    ==8733== Rerun with --leak-check=full to see details of leaked memory
    ==8733== 
    ==8733== For lists of detected and suppressed errors, rerun with: -s
    ==8733== ERROR SUMMARY: 0 errors from 0 contexts (suppressed: 0 from 0)
\end{lstlisting}

С помощью Valgrind обнаружили несколько незначительных ошибок и неосвобождённую память 
после выполнения программы.

\subsection*{Выводы}

Проделав лабораторную работу, я познакомился с полезной утилитой gprof, 
необходимой для измерения времени работы программы и отдельных её частей, 
закрепил навыки работы с утилитой valgrind, а также обнаружил неосвобождённую память в 
своей программе.

\end{document}
