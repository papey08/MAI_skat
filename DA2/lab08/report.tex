\documentclass[12pt]{article}
\usepackage{fullpage}
\usepackage{multicol,multirow}
\usepackage{tabularx}
\usepackage{ulem}
\usepackage[utf8]{inputenc}
\usepackage[russian]{babel}
\usepackage{pgfplots}


\begin{document}

    \section*{Лабораторная работа №\,8 по курсу дискрeтного анализа: 
    Жадные алгоритмы}

    Выполнил студент группы М8О-308Б-20 МАИ \textit{Попов Матвей}.

    \subsection*{Условие}
 
    \begin{enumerate}
    \item Разрабтать жадный алгоритм решения задачи, определяемой своим
    вариантом. Доказать его корректность, оценить скорость и объём
    затрачиваемой оперативной памяти.
    \item \textbf{Вариант 4: Откорм бычков.} Бычкам дают пищевые добавки, чтобы 
    ускорить их рост. Каждая добавка содержит некоторые из $N$ действующих 
    веществ. Соотношения количеств веществ в добавках могут отличаться.
    Воздействие добавки определяется как $c_1a_1 + c_2a_2 + … + c_Na_N$, где 
    $a_i$ — количество $i$-го вещества в добавке, $c_i$ — неизвестный 
    коэффициент, связанный с веществом и не зависящий от добавки. Чтобы найти 
    неизвестные коэффициенты $c_i$, Биолог может измерить воздействие любой 
    добавки, использовав один её мешок. Известна цена мешка каждой из
    $M(M \leq N)$ различных добавок. Нужно помочь Биологу подобрать самый 
    дешевый наобор добавок, позволяющий найти коэффициенты $c_i$.  Возможно, 
    соотношения веществ в добавках таковы, что определить коэффициенты нельзя.
    \end{enumerate}

    \subsection*{Метод решения}

    Жадные алгоритмы применимы в том случае, если принятие наиболее 
    оптимального решения на каждом шаге решения задачи означает наиболее 
    оптимальное решение задачи в целом. К этой задаче можно применить жадный 
    алгоритм, поскольку чтобы её решить, мы должны отобрать ровно $N$ добавок, 
    а значит эти добавки должны быть наиболее дешёвыми. Идея решения в том, 
    чтобы привести матрицу, составленную из соотношений веществ, к ступенчатому 
    виду, при этом наверх продвигать строки, характеризующие наиболее дешёвые 
    добавки, тогда $N$ верхних строк и будут ответом к задаче. Самое главное 
    запомнить, какие коэффициенты были у этих строк изначально.

    \subsection*{Описание программы}

    Программа состоит из одного файла.

    \subsection*{Дневник отладки}

    \begin{enumerate}
    \item Было решено убрать некоторые проверки, вызывавшие неправильный ответ
    \item Изменены типы данных некоторых переменных
    \item Выполнена отладка некоторых функций, обнаружены ошибки в индексах
    матрицы
    \end{enumerate}


    \subsection*{Тест производительности}

    Ниже приведен тест времени работы алгоритма. По оси $X$ — количество 
    добавок, по оси $Y$ — время выполнения алгоритма в мс (меньше — лучше).
    
    \begin{tikzpicture}
        \begin{axis} [
            ymin = 0
        ]
        \addplot coordinates {
            (100,8) (250,68) (500,479)
        };
        \end{axis}
    \end{tikzpicture}

    \begin{tabular}{ | l | l | l | }
        \hline
            Кол-во добавок & Время (в мс) \\ \hline
            100              & 8           \\
            250              & 68          \\
            500              & 479         \\
        \hline
    \end{tabular}

    Тесты подтвердили временную сложность алгоритма — $O(nm^2)$. Бычки довольны.

    \subsection*{Недочёты}

    Пришлось ради удобства прибегнуть к использованию глобальных переменных, 
    что является нежелательной практикой в программировании.

    \subsection*{Выводы}

    Проделав лабораторную работу, познакомился с концепцией жадных алгоритмов, 
    реализовал приведение матрицы к ступенчатому виду и откормил бычков.

\end{document}
